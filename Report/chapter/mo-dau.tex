


Trong chương này chúng tôi muốn giới thiệu mục tiêu nghiên cứu và bố cục của bài báo cáo. Đầu tiên mục tiêu nghiên cứu sẽ mô tả các thành phần liên quan đến tình hình của 18 tỉnh thành đang có dịch. Cuối cùng là bố cục bài báo cáo chúng tôi sẽ nêu rõ tên và trọng tâm của 5 chương.

Bài báo cáo sử dụng các phương pháp thành phần chính để phân loại biến dựa theo bài báo khoa học \cite{ref}
\section*{1. Mục tiêu nghiên cứu}


\begin{description}
	\item[Mô tả] dữ liệu với các thông số về trung bình, phương sai cung cấp các thông tin dịch tễ cơ bản về 18 tỉnh/thành phố đang có dịch bệnh.
	\item[Ứng dụng] thuật toán phân tích thành phần chính để giảm thiểu số chiều dữ liệu dịch tễ các trường hợp xác nhận nhiễm covid-19 đối với 18 tỉnh/thành phố miền Nam và phân tích nhân tố vào dữ liệu để phân cụm các tỉnh có các tính chất tương tự nhau.
\end{description}

\section*{2. Bố cục báo cáo}

Đề tài này bao gồm năm chương với trọng tâm như sau.

\begin{description}
	\item[Chương \ref{chap.0}. Tổng quan cơ sở lý thuyết] tập trung tổng kết có hệ thống một vài lý thuyết đại số tuyến tính và xác suất và chuẩn hóa dữ liệu để thiết lập các thống kê mô tả cũng như thuật toán phân tích thành phần chính và phân tích nhân tố. 
	\item[Chương \ref{chap.1}. Thuật toán phân tích thành phần chính] được dành giải thích và trình bày thuật toán phân tích thành phần chính theo lý thuyết đại số tuyến tính với các định nghĩa trong thống kê. Ngoài ra, một số tiêu chí giảm thiểu số chiều dữ liệu cũng được trình bày để thiết lập thuật toán phân cụm vùng tỉnh/thành phố có bệnh dịch. 
	\item[Chương \ref{chap.2}. Phương pháp phân tích nhân tố] dành trọn vẹn cho việc khảo cứu thuật toán phân tích nhân tố và cách ứng dụng vào dữ liệu dịch bệnh.
	\item[Chương \ref{chap.3}. Thực nghiệm] đầu tiên trình bày tổng quan dữ liệu và các tiêu chuẩn đánh giá tham số đối với các kiểm định. Phần chính yếu nêu các kết quả ứng dụng các thuật toán vào hai loại dữ liệu thứ cấp thể hiện số ca nhiễm hằng ngày và tích lũy đối với các tỉnh/thành phố đang bùng phát dịch.
	\item[Chương \ref{chap.4}. Kết luận và định hướng nghiên cứu] trình bày kết luận và lượng giá về bài báo cáo.
\end{description}









